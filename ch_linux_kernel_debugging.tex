\chapter{Linux内核调试}

\section{Linux内核调试基础}

\section{Oops}
\index{Oops}
Oops是内核编程中比较容易遇到的问题,为了跟多的了解Oops来便于调试,我对Oops提供的信息进行一个总结,以及如何调试Oops。


一个完整的Oops:

\begin{lstlisting}
BUG: unable to handle kernel paging request at 00316b01
IP: [<c05dd045>] netif_receive_skb+0x335/0x377
*pde = 00000000
Thread overran stack, or stack corrupted
Oops: 0000 [#1] SMP
last sysfs file: /sys/block/hda/size
Modules linked in: mymod ipv6 autofs4 nls_utf8 cifs lockd sunrpc dm_multipath 
scsi_dh video output sbs sbshc battery lp sg snd_ens1371 gameport ide_cd_mod 
snd_rawmidi cdrom snd_ac97_codec ac97_bus snd_seq_dummy snd_seq_oss 
snd_seq_midi_event snd_seq snd_seq_device snd_pcm_oss snd_mixer_oss 
parport_pc ac floppy serio_raw snd_pcm button parport rtc_cmos rtc_core 
rtc_lib snd_timer snd pcnet32 mii soundcore snd_page_alloc i2c_piix4 i2c_core 
pcspkr dm_snapshot dm_zero dm_mirror dm_region_hash dm_log dm_mod ata_piix 
libata mptspi mptscsih mptbase scsi_transport_spi sd_mod scsi_mod ext3 jbd 
uhci_hcd ohci_hcd ehci_hcd [last unloaded: mymod]

Pid: 0, comm: swapper Not tainted (2.6.30.9 #1) VMware Virtual Platform
EIP: 0060:[<c05dd045>] EFLAGS: 00010206 CPU: 0
EIP is at netif_receive_skb+0x335/0x377
EAX: 00316ae1 EBX: deb7d600 ECX: 00316ae1 EDX: e2f357c0
ESI: 00000008 EDI: de9a4800 EBP: c9403f40 ESP: c9403f10
 DS: 007b ES: 007b FS: 00d8 GS: 0000 SS: 0068
Process swapper (pid: 0, ti=c9403000 task=c0737320 task.ti=c0779000)
Stack:
 00316ae1 c07777a0 e2f787c0 00000000 00000001 00000008 00000010 deb7d600
 c9403f40 deb7d600 00000000 df5acc58 c9403fb0 e0e61db0 00000000 00000010
 de9a4bb8 de9a4b40 de9a4800 00002000 00000001 00000000 1ea2c822 deb7d600
Call Trace:
 [<e0e61db0>] ? pcnet32_poll+0x347/0x66a [pcnet32]
 [<c041f984>] ? run_rebalance_domains+0x13d/0x3ed
 [<c05df364>] ? net_rx_action+0x6a/0xf4
 [<c0429e2a>] ? __do_softirq+0x94/0x138
 [<c0429d96>] ? __do_softirq+0x0/0x138
 <IRQ> <0> [<c0429d94>] ? irq_exit+0x29/0x2b
 [<c040423b>] ? do_IRQ+0x6d/0x83
 [<c0402e89>] ? common_interrupt+0x29/0x30
 [<c040828a>] ? default_idle+0x5b/0x92
 [<c0401a92>] ? cpu_idle+0x3a/0x4e
 [<c063d84b>] ? rest_init+0x53/0x55
 [<c077f7df>] ? start_kernel+0x293/0x298
 [<c077f06a>] ? i386_start_kernel+0x6a/0x6f

d0 89 45 d8 8b 55 d0 8b 42 20 83 e8 20 89 45 d0 8b 4d d0 <8b> 41 20 0f 18 00 
90 89 c8 83 c0 20 3b 45 d4 75 a4 83 7d d8 00

EIP: [<c05dd045>] netif_receive_skb+0x335/0x377 SS:ESP 0068:c9403f10
CR2: 0000000000316b01
---[ end trace 0330855ac41edfb5 ]---
Kernel panic - not syncing: Fatal exception in interrupt
Pid: 0, comm: swapper Tainted: G      D    2.6.30.9 #1
Call Trace:
 [<c0425ff3>] panic+0x3f/0xdf
 [<c0405644>] oops_end+0x8c/0x9b
 [<c041673a>] no_context+0x10c/0x116
 [<c04168c7>] __bad_area_nosemaphore+0xe0/0xe8
 [<c0416933>] bad_area_nosemaphore+0xd/0x10
 [<c0416aa7>] do_page_fault+0xde/0x1e3
 [<c04169c9>] ? do_page_fault+0x0/0x1e3
 [<c064f38d>] error_code+0x6d/0x74
 [<c061007b>] ? tcp_v4_rcv+0x55b/0x600
 [<c04169c9>] ? do_page_fault+0x0/0x1e3
 [<c05dd045>] ? netif_receive_skb+0x335/0x377
 [<e0e61db0>] pcnet32_poll+0x347/0x66a [pcnet32]
 [<c041f984>] ? run_rebalance_domains+0x13d/0x3ed
 [<c05df364>] net_rx_action+0x6a/0xf4
 [<c0429e2a>] __do_softirq+0x94/0x138
 [<c0429d96>] ? __do_softirq+0x0/0x138
 <IRQ>  [<c0429d94>] ? irq_exit+0x29/0x2b
 [<c040423b>] ? do_IRQ+0x6d/0x83
 [<c0402e89>] ? common_interrupt+0x29/0x30
 [<c040828a>] ? default_idle+0x5b/0x92
 [<c0401a92>] ? cpu_idle+0x3a/0x4e
 [<c063d84b>] ? rest_init+0x53/0x55
 [<c077f7df>] ? start_kernel+0x293/0x298
 [<c077f06a>] ? i386_start_kernel+0x6a/0x6f
\end{lstlisting}

 

 

解析Oops的具体含义:

\begin{lstlisting}
BUG: unable to handle kernel paging request at 00316b01
IP: [<c05dd045>] netif_receive_skb+0x335/0x377
*pde = 00000000
Thread overran stack, or stack corrupted
Oops: 0000 [#1] SMP
last sysfs file: /sys/block/hda/size
Modules linked in: mymod ipv6 autofs4 nls_utf8 cifs lockd sunrpc dm_multipath 
scsi_dh video output sbs sbshc battery lp sg snd_ens1371 gameport ide_cd_mod 
snd_rawmidi cdrom snd_ac97_codec ac97_bus snd_seq_dummy snd_seq_oss 
snd_seq_midi_event snd_seq snd_seq_device snd_pcm_oss snd_mixer_oss 
parport_pc ac floppy serio_raw snd_pcm button parport rtc_cmos rtc_core 
rtc_lib snd_timer snd pcnet32 mii soundcore snd_page_alloc i2c_piix4 i2c_core 
pcspkr dm_snapshot dm_zero dm_mirror dm_region_hash dm_log dm_mod ata_piix 
libata mptspi mptscsih mptbase scsi_transport_spi sd_mod scsi_mod ext3 jbd 
uhci_hcd ohci_hcd ehci_hcd [last unloaded: mymod]
\end{lstlisting}

上面这段这个是载入的模块信息

\begin{lstlisting}
Pid: 0, comm: swapper Not tainted (2.6.30.9 #1) VMware Virtual Platform
EIP: 0060:[<c05dd045>] EFLAGS: 00010206 CPU: 0
EIP is at netif_receive_skb+0x335/0x377
\end{lstlisting}

EIP这行指明发生Oops的具体位置,我们可以通过这个来找到出现Oops的源代码的具体行。

具体方法如下:

通过使用objdump -S反汇编netif\_receice\_skb所在的目标文件,然后找到偏移量为0x355的行,
看看这行是有什么代码汇编来的,再结合寄存器的值就能分析这个Oops的原因了。

 

\begin{lstlisting}
EAX: 00316ae1 EBX: deb7d600 ECX: 00316ae1 EDX: e2f357c0
ESI: 00000008 EDI: de9a4800 EBP: c9403f40 ESP: c9403f10
 DS: 007b ES: 007b FS: 00d8 GS: 0000 SS: 0068
Process swapper (pid: 0, ti=c9403000 task=c0737320 task.ti=c0779000)
Stack:
 00316ae1 c07777a0 e2f787c0 00000000 00000001 00000008 00000010 deb7d600
 c9403f40 deb7d600 00000000 df5acc58 c9403fb0 e0e61db0 00000000 00000010
 de9a4bb8 de9a4b40 de9a4800 00002000 00000001 00000000 1ea2c822 deb7d600
\end{lstlisting}

上面这段是寄存器和栈的信息。

 

\begin{lstlisting}
Call Trace:
 [<e0e61db0>] ? pcnet32_poll+0x347/0x66a [pcnet32]
 [<c041f984>] ? run_rebalance_domains+0x13d/0x3ed
 [<c05df364>] ? net_rx_action+0x6a/0xf4
 [<c0429e2a>] ? __do_softirq+0x94/0x138
 [<c0429d96>] ? __do_softirq+0x0/0x138
 <IRQ> <0> [<c0429d94>] ? irq_exit+0x29/0x2b
 [<c040423b>] ? do_IRQ+0x6d/0x83
 [<c0402e89>] ? common_interrupt+0x29/0x30
 [<c040828a>] ? default_idle+0x5b/0x92
 [<c0401a92>] ? cpu_idle+0x3a/0x4e
 [<c063d84b>] ? rest_init+0x53/0x55
 [<c077f7df>] ? start_kernel+0x293/0x298
 [<c077f06a>] ? i386_start_kernel+0x6a/0x6f
\end{lstlisting}

发生Oops的内核栈信息。

 

\begin{lstlisting}
Code: 74 14 f0 ff 83 a8 00 00 00 8b 4d d8 89 d8 8b 53 14 57 ff 51 08 58 8b 45 
d0 89 45 d8 8b 55 d0 8b 42 20 83 e8 20 89 45 d0 8b 4d d0 <8b> 41 20 0f 18 00 
90 89 c8 83 c0 20 3b 45 d4 75 a4 83 7d d8 00
EIP: [<c05dd045>] netif_receive_skb+0x335/0x377 SS:ESP 0068:c9403f10
CR2: 0000000000316b01
---[ end trace 0330855ac41edfb5 ]---
Kernel panic - not syncing: Fatal exception in interrupt
Pid: 0, comm: swapper Tainted: G      D    2.6.30.9 #1
\end{lstlisting}

如果kernel报告Tainted,说明kernel被损坏了,在"Trainted:"后面最多会有10个字符的提示信息来表示具体的信息。每一位上使用一个字母来表示,如下:

\begin{table}[!hbp]
\caption{Trainted Kernel标志位含义}
\begin{tabularx}{400pt}{r|X}
\hline
第几位 & 含义 \\
\hline
1 & 'G': 所有的模块都是GPL的License。如果有模块缺少MODULE\_LICENSE()或者声明是Proprietary的,则为'P'。\\
2 & 'F': 如果有模块是使用 insmod -f 强制载入的。否则为空。   \\
3 & 'S': 如果Oops发生在SMP的CPU上,但这个型号的CPU还没有被认为是SMP安全的。\\
4 & 'R': 如果有模块是使用 rmmod -f 强制卸载的。否则为空。    \\
5 & 'M': 有CPU报告了Machine Check Exception,否则为空。      \\
6 & 'B': 如果有page-release函数发现一个错误的page或未知的page标志。  \\
7 & 'U': 来自用户空间的程序设置的这个标志位。                \\
8 & 'D': 内核刚刚死掉,比如Oops或者是bug。                   \\
9 & 'A': ACPI表被覆盖。                                      \\
10 & 'W': 之前kernel已经产生过警告。                         \\
\hline
\end{tabularx}
\end{table}

Tainted字符串主要的目的是告诉调试器这个kernel已经不是一个干净的kernel了。如果一个模块在加载了之后又卸载了,Tainted仍然会保持。


\begin{lstlisting}
Call Trace:
 [<c0425ff3>] panic+0x3f/0xdf
 [<c0405644>] oops_end+0x8c/0x9b
 [<c041673a>] no_context+0x10c/0x116
 [<c04168c7>] __bad_area_nosemaphore+0xe0/0xe8
 [<c0416933>] bad_area_nosemaphore+0xd/0x10
 [<c0416aa7>] do_page_fault+0xde/0x1e3
 [<c04169c9>] ? do_page_fault+0x0/0x1e3
 [<c064f38d>] error_code+0x6d/0x74
 [<c061007b>] ? tcp_v4_rcv+0x55b/0x600
 [<c04169c9>] ? do_page_fault+0x0/0x1e3
 [<c05dd045>] ? netif_receive_skb+0x335/0x377
 [<e0e61db0>] pcnet32_poll+0x347/0x66a [pcnet32]
 [<c041f984>] ? run_rebalance_domains+0x13d/0x3ed
 [<c05df364>] net_rx_action+0x6a/0xf4
 [<c0429e2a>] __do_softirq+0x94/0x138
 [<c0429d96>] ? __do_softirq+0x0/0x138
 <IRQ>  [<c0429d94>] ? irq_exit+0x29/0x2b
 [<c040423b>] ? do_IRQ+0x6d/0x83
 [<c0402e89>] ? common_interrupt+0x29/0x30
 [<c040828a>] ? default_idle+0x5b/0x92
 [<c0401a92>] ? cpu_idle+0x3a/0x4e
 [<c063d84b>] ? rest_init+0x53/0x55
 [<c077f7df>] ? start_kernel+0x293/0x298
 [<c077f06a>] ? i386_start_kernel+0x6a/0x6f
\end{lstlisting}
 

 

最后发现一篇调试Oops的专题:

paper on debugging kernel oops or hang <http://mail.nl.linux.org/kernelnewbies/2003-08/msg00347.html> 虽然是针对2.4的,但还是值得一读。


\subsection{使用VMware捕获Linux的串口输出来调试内核的Oops}
Linux的Kernel在产生Oops后会默认情况下把Oops的相关信息打印在控制台上,只有通过控制台才能看到Oops的信息,而且因为受到控制台行数限制,不能完整的看到Oops的信息,这样对调试Oops很麻烦,一种方法使用虚拟机,把串口输出指定到文件,然后再的Linux的控制台消息重定向到串口,这样可以很方便的捕获串口输出,方便调试Oops。

第一步,在VMware中设置串口输出:

Settings -> Hardware -> Add... 添加一个新的串口设备,指定使用文件输出。

第二步,在Linux中对串口进行重定向。修改 /etc/grub.conf 的kernel 行,在行尾加入如下参数:

\begin{lstlisting}
console=ttyS0,115200 console=tty0
\end{lstlisting}

重启,然后测试一下产生一个Oops,看看串口文件,如下,已经有完整的Oops的信息了:

\begin{lstlisting}
Call Trace:
0:0:0:0: [sda] Assuming drive cache: write through
sd 0:0:0:0: [sda] Assuming drive cache: write through
BUG: unable to handle kernel paging request at 00316b01
IP: [<c05dd045>] netif_receive_skb+0x335/0x377
*pde = 00000000
Thread overran stack, or stack corrupted
Oops: 0000 [#1] SMP
last sysfs file: /sys/block/hda/size
Modules linked in: mymod ipv6 autofs4 nls_utf8 cifs lockd sunrpc 
dm_multipath scsi_dh video output sbs sbshc battery lp sg snd_ens1371 
gameport ide_cd_mod snd_rawmidi cdrom snd_ac97_codec ac97_bus 
snd_seq_dummy snd_seq_oss snd_seq_midi_event snd_seq snd_seq_device 
snd_pcm_oss snd_mixer_oss parport_pc ac floppy serio_raw snd_pcm 
button parport rtc_cmos rtc_core rtc_lib snd_timer snd pcnet32 mii 
soundcore snd_page_alloc i2c_piix4 i2c_core pcspkr dm_snapshot dm_zero
dm_mirror dm_region_hash dm_log dm_mod ata_piix libata mptspi mptscsih
mptbase scsi_transport_spi sd_mod scsi_mod ext3 jbd uhci_hcd ohci_hcd 
ehci_hcd [last unloaded: mymod]

Pid: 0, comm: swapper Not tainted (2.6.30.9 #1) VMware Virtual Platform
EIP: 0060:[<c05dd045>] EFLAGS: 00010206 CPU: 0
EIP is at netif_receive_skb+0x335/0x377
EAX: 00316ae1 EBX: deb7d600 ECX: 00316ae1 EDX: e2f357c0
ESI: 00000008 EDI: de9a4800 EBP: c9403f40 ESP: c9403f10
 DS: 007b ES: 007b FS: 00d8 GS: 0000 SS: 0068
Process swapper (pid: 0, ti=c9403000 task=c0737320 task.ti=c0779000)
Stack:
 00316ae1 c07777a0 e2f787c0 00000000 00000001 00000008 00000010 deb7d600
 c9403f40 deb7d600 00000000 df5acc58 c9403fb0 e0e61db0 00000000 00000010
 de9a4bb8 de9a4b40 de9a4800 00002000 00000001 00000000 1ea2c822 deb7d600
Call Trace:
 [<e0e61db0>] ? pcnet32_poll+0x347/0x66a [pcnet32]
 [<c041f984>] ? run_rebalance_domains+0x13d/0x3ed
 [<c05df364>] ? net_rx_action+0x6a/0xf4
 [<c0429e2a>] ? __do_softirq+0x94/0x138
 [<c0429d96>] ? __do_softirq+0x0/0x138
 <IRQ> <0> [<c0429d94>] ? irq_exit+0x29/0x2b
 [<c040423b>] ? do_IRQ+0x6d/0x83
 [<c0402e89>] ? common_interrupt+0x29/0x30
 [<c040828a>] ? default_idle+0x5b/0x92
 [<c0401a92>] ? cpu_idle+0x3a/0x4e
 [<c063d84b>] ? rest_init+0x53/0x55
 [<c077f7df>] ? start_kernel+0x293/0x298
 [<c077f06a>] ? i386_start_kernel+0x6a/0x6f
Code: 74 14 f0 ff 83 a8 00 00 00 8b 4d d8 89 d8 8b 53 14 57 ff 51 08 58 
8b 45 d0 89 45 d8 8b 55 d0 8b 42 20 83 e8 20 89 45 d0 8b 4d d0 <8b> 41 
20 0f 18 00 90 89 c8 83 c0 20 3b 45 d4 75 a4 83 7d d8 00
EIP: [<c05dd045>] netif_receive_skb+0x335/0x377 SS:ESP 0068:c9403f10
CR2: 0000000000316b01
---[ end trace 0330855ac41edfb5 ]---
Kernel panic - not syncing: Fatal exception in interrupt
Pid: 0, comm: swapper Tainted: G      D    2.6.30.9 #1
Call Trace:
 [<c0425ff3>] panic+0x3f/0xdf
 [<c0405644>] oops_end+0x8c/0x9b
 [<c041673a>] no_context+0x10c/0x116
 [<c04168c7>] __bad_area_nosemaphore+0xe0/0xe8
 [<c0416933>] bad_area_nosemaphore+0xd/0x10
 [<c0416aa7>] do_page_fault+0xde/0x1e3
 [<c04169c9>] ? do_page_fault+0x0/0x1e3
 [<c064f38d>] error_code+0x6d/0x74
 [<c061007b>] ? tcp_v4_rcv+0x55b/0x600
 [<c04169c9>] ? do_page_fault+0x0/0x1e3
 [<c05dd045>] ? netif_receive_skb+0x335/0x377
 [<e0e61db0>] pcnet32_poll+0x347/0x66a [pcnet32]
 [<c041f984>] ? run_rebalance_domains+0x13d/0x3ed
 [<c05df364>] net_rx_action+0x6a/0xf4
 [<c0429e2a>] __do_softirq+0x94/0x138
 [<c0429d96>] ? __do_softirq+0x0/0x138
 <IRQ>  [<c0429d94>] ? irq_exit+0x29/0x2b
 [<c040423b>] ? do_IRQ+0x6d/0x83
 [<c0402e89>] ? common_interrupt+0x29/0x30
 [<c040828a>] ? default_idle+0x5b/0x92
 [<c0401a92>] ? cpu_idle+0x3a/0x4e
 [<c063d84b>] ? rest_init+0x53/0x55
 [<c077f7df>] ? start_kernel+0x293/0x298
 [<c077f06a>] ? i386_start_kernel+0x6a/0x6f
\end{lstlisting}


\subsection{Kdump}
\index{kdump}

Kdump是一种调试Linux内核的方法,用于在Linux内核出现Oops之后自动dump内核映像到指定位置的机制,编译我们的事后调试。

Kdump is a new kernel crash dumping mechanism and is very reliable. The crash dump is
captured from the context of a freshly booted kernel and not from the context of the crashed
kernel. Kdump uses kexec to boot into a second kernel whenever the system crashes. This
second kernel, often called a capture kernel, boots with very little memory and captures the
dump image.

1. Install kexec-tools

\begin{lstlisting}
yum install kexec-tools
\end{lstlisting}

2. write kdump config file /etc/kdump.conf with following content.

\begin{lstlisting}
path /var/crash
core_collector makedumpfile -d 31 -c
\end{lstlisting}

3. change /etc/grub.conf append crashkernel=128M@16M to the end of kernel line. example:

\begin{lstlisting}
default=0
timeout=5
splashimage=(hd0,0)/grub/splash.xpm.gz
hiddenmenu
title Red Hat Enterprise Linux Server (2.6.18-128.el5)
        root (hd0,0)
        kernel /vmlinuz-2.6.18-128.el5 ro root=/dev/mapper/luks-10d5d533-38f6-482c-982d-bfb488adfbed
 rhgb quiet crashkernel=128M@16M
        initrd /initrd-2.6.18-128.el5.img
\end{lstlisting}

4. post config

\begin{lstlisting}
chkconfig kdump on
service kdump start
reboot
\end{lstlisting}

系统重新启动后,kdump就会生效,之后系统如果再次出现crash,crash文件会存放在\emph{/var/crash/}。


\subsection{crash文件的分析}

netdump产生的crash文件可以使用crash来进行分析。

分析方法如下:

\section{总结}

Crash Utility White Paper\cite{crash-white-paper}。

