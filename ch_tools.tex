%# -*- coding: utf-8 -*-

\chapter{调试程序的工具}

\section{GDB}

\section{strace}
\index{strace}

\subsection{简介}
\shcmd{strace}~能跟踪一个程序的系统调用情况。它会动态显示一个程序执行系统调用的情况,
它会输出每个系统调用的返回值,如果返回值不是0的话还会显示errno的信息。

\subsection{用法}
\shcmd{strace}~的用法很简单,只需用在你要执行的程序前面加上strace来执行即可。

\begin{lstlisting}[language={sh}]
strace prog args...
\end{lstlisting}

\shcmd{strace}~能用于跟踪某些程序异常退出,它能告诉你程序是因为执行什么系统调用退出的。
\shcmd{strace}~还能用于分析程序的挂起状态,
当一个程序挂起不响应请求时可以使用~\shcmd{strace}~看看程序内部在
执行什么\footnote{也可以使用~\shcmd{gdb}~来分析程序挂起的原因,
即在程序挂起的时候通过~\shcmd{gdb}~获得程序的栈。}。

\subsection{示例}
使用~\shcmd{strace}~来跟踪命令~\shcmd{ls \-l \*cpp}~的执行过程:\\
\begin{lstlisting}[language={sh}]
$ strace /bin/ls -l *cpp
execve("/bin/ls", ["/bin/ls", "-l", "segv_handler.cpp", "test_mprotect.cpp", "test_signal.cpp"], [/* 42 vars */]) = 0
brk(0)                                  = 0x8d3b000
...
...
...
lstat64("test_mprotect.cpp", {st_mode=S_IFREG|0644, st_size=2002, ...}) = 0
getxattr("test_mprotect.cpp", "system.posix_acl_access"..., 0x0, 0) = -1 EOPNOTSUPP (Operation not supported)
lstat64("test_signal.cpp", {st_mode=S_IFREG|0664, st_size=5200, ...}) = 0
getxattr("test_signal.cpp", "system.posix_acl_access"..., 0x0, 0) = -1 EOPNOTSUPP (Operation not supported)
fstat64(1, {st_mode=S_IFCHR|0620, st_rdev=makedev(136, 8), ...}) = 0
mmap2(NULL, 4096, PROT_READ|PROT_WRITE, MAP_PRIVATE|MAP_ANONYMOUS, -1, 0) = 0xb7d1e000
open("/etc/localtime", O_RDONLY)        = 3
fstat64(3, {st_mode=S_IFREG|0644, st_size=163, ...}) = 0
fstat64(3, {st_mode=S_IFREG|0644, st_size=163, ...}) = 0
mmap2(NULL, 4096, PROT_READ|PROT_WRITE, MAP_PRIVATE|MAP_ANONYMOUS, -1, 0) = 0xb7d1d000
read(3, "TZif\0\0\0\0\0\0\0\0\0\0\0\0\0\0\0\0\0\0\0\4\0\0\0\4\0\0\0\0"..., 4096) = 163
close(3)                                = 0
munmap(0xb7d1d000, 4096)                = 0
clock_gettime(CLOCK_REALTIME, {1265699498, 844189000}) = 0
stat64("/etc/localtime", {st_mode=S_IFREG|0644, st_size=163, ...}) = 0
write(1, "-rw-rw-r--  1 cm cm 4554 Feb  9 "..., 55-rw-rw-r--  1 cm cm 4554 Feb  9 11:52 segv_handler.cpp
) = 55
stat64("/etc/localtime", {st_mode=S_IFREG|0644, st_size=163, ...}) = 0
write(1, "-rw-r--r--  1 cm cm 2002 Jan 19 "..., 56-rw-r--r--  1 cm cm 2002 Jan 19 11:36 test_mprotect.cpp
) = 56
stat64("/etc/localtime", {st_mode=S_IFREG|0644, st_size=163, ...}) = 0
write(1, "-rw-rw-r--  1 cm cm 5200 Feb  9 "..., 54-rw-rw-r--  1 cm cm 5200 Feb  9 11:49 test_signal.cpp
) = 54
exit_group(0)                           = ?
\end{lstlisting}

\section{GLIBC~提供的内存错误检查机制}
\subsection{简介}
GLIBC~自~2.x~自带了部分内存检测机制。
它能检查一些简单内存错误,包括:
\begin{itemize}
\item 重复释放(double free);
\item 越界写(off-by-one bugs)
\end{itemize}

通过设置环境变量~MALLOC\_CHECK\_~为不同的值可以控制~GLIBC~的内存错误检测级别:

\begin{description}
\item[0] 忽略检查到的错误。
\item[1] 检测到错误时向屏幕输出错误信息,但不中止程序。
\item[2] 检测到错误时不向屏幕输出错误信息,直接调用~\code{abort()}~中止程序。
\item[3] 检测到错误时向屏幕输出错误信息,同时调用~\code{abort()}~中止程序。
\end{description}

通过设置~MALLOC\_CHECK\_~为~2~或者~3~可以及时检测到内存错误,
否则的话某些内存错误要很久才会暴露,那个时候再查找错误将是件很困难的事。

示例:使用~MALLOC\_CHECK\_~来检测使用重复释放内存的问题。

\begin{lstlisting}
//
// double_free.cpp
//

void foo(char* x)
{
    delete[] x;
}


int main()
{
    char* x = new char[20];

    foo(x);

    for (int i=0; i<20; ++i) {
        x[i] = 'a' + i;
    }

    delete[] x;

    return 0;
}
\end{lstlisting}

上面这段代码存在重复释放内存的错误,
我们看看~MALLOC\_CHECK\_~分别为~0~和~3~的效果。

\begin{lstlisting}
$ g++ -g double_free.cpp
$ export MALLOC_CHECK_=0
$ ./a.out
$ export MALLOC_CHECK_=3
$ ./a.out
*** glibc detected *** ./a.out: free(): invalid pointer: 0x08284008 ***
======= Backtrace: =========
/lib/tls/i686/cmov/libc.so.6[0xd2fff1]
/lib/tls/i686/cmov/libc.so.6(cfree+0xd6)[0xd34836]
/usr/lib/libstdc++.so.6(_ZdlPv+0x21)[0x47c6f1]
/usr/lib/libstdc++.so.6(_ZdaPv+0x1d)[0x47c74d]
./a.out[0x8048591]
/lib/tls/i686/cmov/libc.so.6(__libc_start_main+0xe6)[0xcdbb56]
./a.out[0x8048481]
======= Memory map: ========
003c4000-004aa000 r-xp 00000000 08:02 554348     /usr/lib/libstdc++.so.6.0.13
004aa000-004ae000 r--p 000e6000 08:02 554348     /usr/lib/libstdc++.so.6.0.13
004ae000-004af000 rw-p 000ea000 08:02 554348     /usr/lib/libstdc++.so.6.0.13
004af000-004b6000 rw-p 00000000 00:00 0
0091c000-0091d000 r-xp 00000000 00:00 0          [vdso]
00a86000-00aa1000 r-xp 00000000 08:02 166175     /lib/ld-2.10.1.so
00aa1000-00aa2000 r--p 0001a000 08:02 166175     /lib/ld-2.10.1.so
00aa2000-00aa3000 rw-p 0001b000 08:02 166175     /lib/ld-2.10.1.so
00b92000-00bb6000 r-xp 00000000 08:02 195080     /lib/tls/i686/cmov/libm-2.10.1.so
00bb6000-00bb7000 r--p 00023000 08:02 195080     /lib/tls/i686/cmov/libm-2.10.1.so
00bb7000-00bb8000 rw-p 00024000 08:02 195080     /lib/tls/i686/cmov/libm-2.10.1.so
00cc5000-00e03000 r-xp 00000000 08:02 195076     /lib/tls/i686/cmov/libc-2.10.1.so
00e03000-00e04000 ---p 0013e000 08:02 195076     /lib/tls/i686/cmov/libc-2.10.1.so
00e04000-00e06000 r--p 0013e000 08:02 195076     /lib/tls/i686/cmov/libc-2.10.1.so
00e06000-00e07000 rw-p 00140000 08:02 195076     /lib/tls/i686/cmov/libc-2.10.1.so
00e07000-00e0a000 rw-p 00000000 00:00 0
00f26000-00f42000 r-xp 00000000 08:02 166231     /lib/libgcc_s.so.1
00f42000-00f43000 r--p 0001b000 08:02 166231     /lib/libgcc_s.so.1
00f43000-00f44000 rw-p 0001c000 08:02 166231     /lib/libgcc_s.so.1
08048000-08049000 r-xp 00000000 08:04 239702     /home/cm/sandbox/a.out
08049000-0804a000 r--p 00000000 08:04 239702     /home/cm/sandbox/a.out
0804a000-0804b000 rw-p 00001000 08:04 239702     /home/cm/sandbox/a.out
08284000-082a5000 rw-p 00000000 00:00 0          [heap]
b77e0000-b77e2000 rw-p 00000000 00:00 0
b77f7000-b77f9000 rw-p 00000000 00:00 0
bfeb1000-bfec6000 rw-p 00000000 00:00 0          [stack]
Aborted (core dumped)
\end{lstlisting}

在设置为~0~时,不会检测到错误,程序看起来正常。
在设置为~3~时,在第二次释放内存立即检测到了错误,
向屏幕输出了错误信息,并立即调用~\code{abort()}~终止了程序。


\section{mcheck} \index{mcheck}
\subsection{简介}
\subsection{使用mcheck来检查内存错误}
mcheck~是~GLIBC~的一个扩展,它能提供内存的错误
\subsection{使用mtrace来跟踪内存泄露}

\section{YAMD - Yet Another Malloc Debugger}

\label{yamd}
\index{YAMD}

C和C++程序要自己来管理内存的分配与释放,这就会出现内存的非法访问。

要调试内存的非法访问是一件比较费时间的事,
于是乎就出现了各种各样的内存调试工具,
现在已经有很多针对mall/free new/delete的调试工具,
有商业的工具比如purify,也有许多开源的工具比如valgrand,
但各有利弊,有的需要对源代码进行修改(凡是这种工具这种我就不考虑了),
有的需要在链接的时候进行插桩,比如purify。

在众多工具中我发现YAMD(Yet Another Malloc Debugger)其实是一个不错的选择,
它可以不需要我们现有的程序做任何改动就能发现程序的错误。

能支持YAMD的平台:Linux, DJGPP。 它能处理C的malloc/free及C++的new/delete

\subsection{原理}

YAMD可以通过设置环境变量LD\_PRELOAD来先于libc加载,从而实现自己的malloc系列函数。

\subsection{使用YAMD}

要使用~YAMD,需要你的程序使用~GCC~进行编译,需要~\code{-g}~选项,
而且链接时不能有~\code{-fomit-frame-pointer}~和~\code{-s}~参数。
也可以静态链接YAMD,在此就不做讨论了。

YAMD~提供了一个脚本rum-yamd用来运行你要调试的程序
rum-yamd [options] program [args]

示例:
\begin{lstlisting}
run-yamd -l 2 -o x.log ./foo
\end{lstlisting}

\subsection{更多信息}

  YAMD的主页
    https://www.cs.hmc.edu/~nate/yamd/

  Options of run-yamd:

  -r : When corrupted magic bytes are found, fix ("repair") them after
   reporting to prevent further reports.  Default is to leave them.

  -d : If corrupted magic bytes are found, abort the program after
   reporting ("die").  Default is to continue.

  -f : Check the front (prefix) of blocks, instead of the end.  This
   is useful for catching negative overruns.

  -a nnn : Set default alignment of allocated blocks for which
   alignment is not specified (as with `memalign').  Must be a power
   of 2, and should be as small as possible to catch the most bugs.
   Default is 1, which provides the best checking, and is also faster
   (it avoids a lot of manual overrun checking).  The unaligned memory
   accesses have not proven to be a problem.

  -o file : Direct the log output to `file'.  Default is stderr.

  -l n : Set the minimum log level to n.  1 is INFO, 2 is WARNING, 3
   is ERROR.

  -s : Inform run-yamd that the program in question has YAMD
   statically linked with it, to prevent it from loading it
   dynamically (which is the default).

  -i : When loading YAMD dynamically, have other programs exec'd by
   the child inherit YAMD.  This is not well tested and may fail if
   the grandchildren were linked against a different libc version, for
   instance.

  -c file : Specify the YAMD shared object file.  Default is
   LIBDIR/libyamd-dynamic.so.

  -n : Omit the step which symifies the log (it can be slow).  This is
   probably not useful except for testing.

  -v : Print version and exit.

  -h : Print short help message and exit.


\section{valgrind}
\label{valgrind} \index{valgrind}
\subsection{简介}
Valgrind is an award-winning instrumentation framework for building dynamic analysis tools. There are Valgrind tools that can automatically detect many memory management and threading bugs, and profile your programs in detail.

https://www.valgrind.org

\subsection{使用}

更多关于valgrind的信息请参考valgrind的手册\cite{valgrind-man}

\section{Purify}
\label{Purify} \index{Purify}
Purify是IBM的PurifyPlus\index{PurifyPlus}套件中的分析程序内存错误的工具。
Purify支持的平台包括Windows和Linux,HP-UX,AIX等UNIX操作系统。


\section{Bounds Checking}
\emph{Bounds checking}~是~GCC~的一个补丁,
它通过增加一个新的编译选项~\code{-fbounds-checking}~
它能实现运行时的指针和数组的边界检查。

\subsection{使用示例}

\subsubsection{检查指针越界写}

\begin{lstlisting}
$ cat bc.c
     1  main() {
     2      char *malloc(); char *a = malloc(2);
     3      a[2] = 'x';
     4  }
$ gcc -fbounds-checking bc.c
$ export GCC_BOUNDS_OPTS="-print-heap"
$ ./a.out
Bounds Checking GCC v gcc-3.4.6-3.2 Copyright (C) 1995 Richard W.M. Jones
Bounds Checking comes with ABSOLUTELY NO WARRANTY. For details see file
`COPYING' that should have come with the source to this program.
Bounds Checking is free software, and you are welcome to redistribute it
under certain conditions. See the file `COPYING' for details.
For more information, set GCC_BOUNDS_OPTS to `-help'
bc.c:3:Bounds error: attempt to reference memory overrunning the end of an object.
bc.c:3:  Pointer value: 0x808a002, Size: 1
bc.c:3:  Object `malloc':
bc.c:3:    Address in memory:    0x808a000 .. 0x808a001
bc.c:3:    Size:                 2 bytes
bc.c:3:    Element size:         1 bytes
bc.c:3:    Number of elements:   2
bc.c:3:    Created at:           bc.c, line 2
bc.c:3:    Storage class:        heap
Aborted (core dumped)
\end{lstlisting}


\subsection{如何编译Bounds checking的补丁}

Bounds checking~的补丁当前支持~GCC~4.0.4,4.0.2,3.4.6。
可以从~https://williambader.com/bounds/example.html~下载。
下载完了按照它的说明进行编译即可。
需要说明的时候在configure的时候可以指定选项,
下面的configure命令式我使用的,
我只需要对~C,C++~语言的支持,
而且在编译后的~gcc~可执行文件后面加上3.4.6的后缀,
便于我区分这个~gcc~是不是有~Bounds checking的。

\begin{lstlisting}
../configure --prefix=/home/cm/opt \
                      --enable-languages=c,c++ \
                      --enable-shared   \
                      --disable-werror  \
                      --enable-threads=posix    \
                      --program-suffix=3.4.6    \
                      --without-included-gettext
\end{lstlisting}

更多关于~Bounds checking~的信息参考~\cite{bounds-checking-example}。


\section{抓包工具}

\subsection{tcpdump}
tcpdump\index{tcpdump}是最原始的,也是最通用的抓包的工具,在很多系统下都有找到它。

\begin{lstlisting}
tcpdump -i any -s 0 -w sip.pcap "port 5060"
\end{lstlisting}

\subsection{Wireshark}
\label{sec:wireshark}
\index{Wireshark}

Wireshark~是图形化的抓包的工具,它的前身是~ethereal。
没什么好说的,图形化的界面,很容易上手。

在~Wireshark~中提供部分的统计功能,在分析大量包的统计特性时比较有用,
比如分析丢包率,各种消息的比例等。

Wireshark~是可扩展的,可以为它添加新的协议解析插件。

\subsection{tshark}

tshark~比是Wireshark的命令行版本,它能像tcpdump一样用于抓包和对包进行分析。
tshark~提供更多的功能,比如它对过滤条件支持更丰富,它还能提供统计的功能。

\subsubsection{tshark~的用法}

tshark~的用法如下:

\begin{lstlisting}
tshark [ -a <capture autostop condition> ] ...
  [ -b <capture ring buffer option>] ...
  [ -B <capture buffer size (Win32 only)> ]
  [ -c <capture packet count> ]
  [ -C <configuration profile> ]
  [ -d <layer type>==<selector>,<decode-as protocol> ]
  [ -D ]
  [ -e <field> ]
  [ -E <field print option> ]
  [ -f <capture filter> ]
  [ -F <file format> ]
  [ -h ]
  [ -i <capture interface>|- ]
  [ -K <keytab> ]
  [ -l ]
  [ -L ]
  [ -n ]
  [ -N <name resolving flags> ]
  [ -o <preference setting> ] ...
  [ -p ] [ -q ]
  [ -r <infile> ]
  [ -R <read (display) filter> ]
  [ -s <capture snaplen> ]
  [ -S ]
  [ -t ad|a|r|d|dd|e ]
  [ -T pdml|psml|ps|text|fields ]
  [ -v ] [ -V ]
  [ -w <outfile>|- ]
  [ -x ]
  [ -X <eXtension option>]
  [ -y <capture link type> ]
  [ -z <statistics> ]
  [ <capture filter> ]
\end{lstlisting}

主要的命名行参数如下:

\noindent
\shcmd{-c  <\shparam{capture packet count}>}

\paramdesc{设置抓包总共数目的限制。如果是从文件读取的数据,最多读取指定数目的包。}

\noindent
\shcmd{-d  <\shparam{layer type}>==<\shparam{selector}>,<\shparam{decode-as protocol}>}

\paramdesc{类似~Wireshark~的“Decode As...”,强制把符合条件包解析为指定的协议。
比如~\shcmd{-d tcp.port==8888,http}~会强制把~TCP~的~8888~端口上数据解析为~HTTP~协议。
\shcmd{-d .}~列出所有支持的~\shparam{selector}。}


\noindent
\shcmd{-D}

\paramdesc{打印出可用的网络接口。该命令的输出可以供~\shcmd{-i}~使用。}

\noindent
\shcmd{-f  <\shparam{capture filter}>}

\paramdesc{指定对包的过滤条件。该条件通tcpdump抓包时使用的过滤条件一致。}

\noindent
\shcmd{-F  <\shparam{file format}>}

\paramdesc{在要把抓获的包写入文件时(使用参数~\shcmd{-w}),指定数据存储的文件格式。
如果没有指定~\shparam{file format},打印所有支持的文件格式。}


\noindent
\shcmd{-i  <\shparam{capture interface}>|\shparam{-}}

\paramdesc{设置抓包的网络的接口,可以使用命令~\shcmd{tshark -D}~获得当前系统可用的网络接口。
如果没有指定该参数,使用第一个非~loopback~接口。
如果要抓取所有接口上包,可以使用~\shparam{any}~来代表所有的接口。
如果参数是~\shparam{-},表示从标准输入读取数据,一般用于命令管道。}

\noindent
\shcmd{-n}

\paramdesc{禁用名字解析(DNS、TCP/UDP端口名等)。
参数~\shcmd{-N}~会覆盖该参数。见下。}

\noindent
\shcmd{-N  <\shparam{name resolving flags}>}

\paramdesc{指定需要进行名字解析的标志,
以决定在抓包时是否解析IP地址,端口,MAC地址等。
该标志为一个字符串,每个字符表示相应的标志。
支持的标志参见表~\ref{tab:tshark_Nflags}。
如果同时指定参数~\shcmd{-n}~和参数~\shcmd{-N},
参数~\shcmd{-N}~会覆盖参数~\shcmd{-n}。
如果~\shcmd{-N}~和~\shcmd{-n}~都没有指定,开启所有的名字解析。
}

\begin{table}[!bhp]
\begin{tabularx}{400pt}{l|X}
\hline
\hline
标志 & 说明\\
\hline
m & 启用~MAC~地址解析。\\
n & 启用网络地址解析。\\
t & 启用端口名称解析(端口名称定义在/etc/services中)。\\
C & 启用异步DNS解析。\\
\hline
\hline
\end{tabularx}
\caption{tshark名字解析支持的标志}\label{tab:tshark_Nflags}
\end{table}


\noindent
\shcmd{-r  <\shparam{infile}>}

\paramdesc{从指定的文件~\shparam{infile}~中读取抓包文件(支持GZIP压缩过的文件)。}

\noindent
\shcmd{-R  <\shparam{read (display) filter}>}

\paramdesc{使用~Wireshark~的显示过滤条件来作为抓包的过滤条件(比如''sip || tcp.port == 2944'')。
而且在打印包的内容时会先使用过滤条件进行解包,
比如~\shcmd{-R ''\shparam{sip}''}~就比~\shcmd{''\shparam{port 5060}''}~更易读。
默认的是使用~tcpdump~格式的抓包过滤条件。
}

\noindent
\shcmd{-s  <\shparam{capture snaplen}>}

\paramdesc{设置缺省的抓包时每个包的最大长度。
如果设置为~0~表示没有限制,整个包都会被抓取,这是缺省设置。}

\noindent
\shcmd{-S}

\paramdesc{即使在指定了~\shcmd{-w}~的时候也在抓包时实时解包并将其结果向屏幕输出。
默认情况下如果指定了~\shcmd{-w}~参数,抓包的结果会直接写入文件而不向屏幕输出。}

\noindent
\shcmd{-w  <\shparam{outfile}>|-}

\paramdesc{把抓包的结果写入文件~\shparam{outfile},如果指定为‘-’表示写向标准输出。}

\noindent
\shcmd{-z  <\shparam{statistics}>}

\paramdesc{收集统计信息。
如果只想查看统计信息而对每个具体的包不感兴趣的话需要加上~\shcmd{-q}~参数。
常用的统计参数~\shparam{statistics}~如下:}

\noindent
\shcmd{-z io,stat,\shparam{interval}[,\shparam{filter}][,\shparam{filter}]\ldots}

\paramdesc{获取指定时间间隔~\shparam{intervals}~(单位为秒)的IO统计信息。
如果~\shparam{intervals}~为~0,统计所有的包。
参数~\shparam{filter}~为要统计的包过滤条件。
比如''-z io,stat,1,smb\&\&ip.addr==1.2.3.4''产生符合条件''smb\&\&ip.addr==1.2.3.4''的1秒间隔的统计信息。}

\noindent
\shcmd{-z rtp,streams}

\paramdesc{产生所有~RTP~流的统计信息。}

\noindent
\shcmd{-z smb,rtt[,\shparam{filter}]}

\paramdesc{统计~SMB~协议的~RTT(Round-Trip Time)信息。}


\noindent
\shcmd{-z megaco,rtd[,\shparam{filter}]}

\paramdesc{产生~MEGACO~的~RTD~(Response Time Delay)统计信息。
\shparam{filter}~为附加的过滤条件。}


\noindent
\shcmd{-z h225,counter[,\shparam{filter}]}

\paramdesc{统计~H.225~消息和它的原因。}

\noindent
\shcmd{-z h225,srt[,\shparam{filter}]}

\paramdesc{收集~H.225~的~SRT(Service Response Time)统计数据。}

\noindent
\shcmd{-z sip,stat[,\shparam{filter}]}

\paramdesc{统计~SIP~消息。
示例:\shcmd{-z "sip,stat\shparam{,ip.addr==1.2.3.4}"}。}


\subsubsection{tshark~的用法示例}
获取可用的网络接口:
\begin{lstlisting}
tshark -D
\end{lstlisting}


抓所有网络接口上的包:
\begin{lstlisting}
tshark -i any -s 0 -w x.pcap
\end{lstlisting}

\begin{lstlisting}
tshark -i any -s 0 -w sip.pcap -R "sip"
\end{lstlisting}


过滤所有奇数端口的UDP包:
\begin{lstlisting}
tshark -n -R "udp and (udp.port&1)"
\end{lstlisting}

过滤所有偶数端口的UDP包:
\begin{lstlisting}
tshark -n -R "udp and (not (udp.port&1))"
\end{lstlisting}

\shcmd{tshark}~很适合用于脚本中来对包进行分析。

\subsubsection{案例:使用~\shcmd{tshark}~对PoC的RTCP进行统计}
假设我们已经抓取的~RTCP~包经过过滤后都是POC1的floor control包,
因为端口的随机性,默认情况下~\shcmd{tshark}~并不知道这些包是~RTCP~包,
这就需要先对这些包进行强制解包,然后再根据包的类型来统计。
强制解包需要需要使用~\shparam{-d}~参数,这个参数可以重复多次。
下面是一个示例脚本:

\begin{lstlisting}
#!/bin/sh
# Date: Jan 28, 2010
# Author: Chen Ming
# Description:
#   Decode all packets as RTCP. Then make POC1 floor
#   statistics.
#
# Usage:
#   $0 <infile>
#

stat()
{
    key=$1
    file=$2
    count=$(grep "$key" $file | wc -l)
    echo "$key: $count"
}


infile=$1
outfile=rtcp.out

ports=$(tshark -r $infile  | awk '{print $9}' | sort | uniq)
for p in $ports
do
  d="$d -d udp.port==$p,rtcp"
done

tshark $d -r $infile > $outfile

stat "Grant" $outfile
stat "Idle" $outfile
stat "Deny" $outfile
stat "Taken" $outfile
stat "Sender Report" $outfile
\end{lstlisting}

使用示例:
\begin{lstlisting}
./floor_stat.sh  rtcp.pcap
\end{lstlisting}
% TODO: 添加输出结果


\subsection{nettl - HP-UX下的抓包工具}
HP-UX~提供了自己的抓包工具~\emph{nettl}\index{nettl}。
在tcpdump不能用的时候它(比如tcpdump在HP-UX上就不能抓回环上的包)
就能派上用场了。

总结一下nettl的主要用法。

开始抓包:
\begin{lstlisting}
# nettl -tn all -e all -maxtrace 99999 -f /tmp/tix
# nettl -tn loopback -e ns_ls_tcp -maxtrace 99999 -f /tmp/tix
# nettl -tn pduin pduout -e ns_ls_loopback -tm 100000 -f /tmp/local
# nettl -tn pduin pduout -e ns_ls_loopback -m 56 -tm 100000 -f /tmp/local
# nettl -tn pduin -e ns_ls_loopback -m 56 -tm 100000 -f /tmp/local
\end{lstlisting}

-m size 限制每个包的大小。我们不一定对所有的包都感兴趣,在只对包头干兴趣的时候这个选项就非常有效。
FDB协议使用的包头为16个字节,再加上IP头的20个字节,TCP头的20个字节,也就是我们只需要总共56个字节就能确定一个包的基本信息了。

-e subsytem 要抓包的类型,可以使用 nettl -status 来获取。常用的subsystem有:
\begin{lstlisting}
ns_ls_loopback
ns_ls_ip
ns_ls_tcp
ns_ls_udp
ns_ls_icmp
\end{lstlisting}

-tm maxsize 每个文件的最大大小,如果超过此大小,会使用下一个抓包文件。单位:KB。有效值:100~99999

抓包的输出文件为 /tmp/tix.xxxx,使用 ls -l /tmp/tix.* 来检查

说明:
在 ns\_ls\_loopback 上抓包如果指定了pduin 和 pduout 每个包会抓到2份,因为一进一出就是两份。
如果指定 -tn all -e all 一个包也会抓到多份,因为一个包可能属于不同的~subsystem,比如一个tcp包既属于tcp,也属于ip等。

抓的包可以使用wireshark来打开并进行分析。也可以使用HP-UX自带的netfmt来分析。

查看状态及-entity可用的信息:
\begin{lstlisting}
# nettl -status
\end{lstlisting}

停止抓包:
\begin{lstlisting}
# nettl -tf -e all
\end{lstlisting}

\subsubsection{对包的分析}

我们可以使用~netfmt~来分析捕获的包:

得到包的统计信息:
\begin{lstlisting}
netfmt -s /tmp/nettl_t*
\end{lstlisting}

解析包的内容:
\begin{lstlisting}
netfmt -N -l -f /tmp/nettl_t* | more
\end{lstlisting}

可以过滤我们感兴趣的包,使用~-c~来传入过滤文件
\begin{lstlisting}
netfmt -N -l -c filter -f /tmp/nettl_t* | more
\end{lstlisting}

filter~为过滤文件,文件内容的类似如下:
\begin{lstlisting}
filter tcp_sport 1234
filter tcp_dport 1234
\end{lstlisting}
每一行为一个过滤条件,行与行之间是\emph{或}的关系。
常用的过滤条件有

\noindent
\shcmd{filter dest \param{value}}\\
\shcmd{filter source \param{value}}

\paramdesc{\param{value}~是以16进制表示的6个字节的硬件~MAC~地址(不要开头的0x),可以使用'-'来作为分割符。示例:00-24-8C-63-A4-83。 }

\noindent
\shcmd{filter dsap \param{value}}\\
\shcmd{filter ssap \param{value}}

\paramdesc{\param{value}~可以是一个以0x开头的16进制数,
也可以是一个以0开头的八进制数,或十进制的数。范围为0到255。}

\noindent
\shcmd{filter interface \shparam{value}}

\paramdesc{\shparam{value}~是网络接口的名字,lan\shparam{n}~用于表示~LAN~接口,
其中~\shparam{n}~是序号,比如第一个网络接口名称为~lan0。
lon~表示~loopback~接口。}

\noindent
\shcmd{filter ip\_daddr \param{value}}\\
\shcmd{filter ip\_saddr \param{value}}

\paramdesc{\param{value}~是主机名或点分十进制格式的IP地址。
示例:www.example.com 或 1.2.3.4。}

\noindent
\shcmd{filter ip6\_daddr \param{value}}\\
\shcmd{filter ip6\_saddr \param{value}}

\paramdesc{\param{value}~是一个主机名或以冒号分割的IPv6地址。
示例:www.example.com 或 2000:1234:5678:abcd::13f2。}

\noindent
\shcmd{filter ip\_proto \param{value}}\\
\shcmd{filter ip6\_proto \param{value}}

\paramdesc{\param{value}~可以是一个以0x开头的16进制数,
也可以是一个以0开头的八进制数,或十进制的数。范围为0到255。}

\noindent
\shcmd{filter tcp\_dport \param{value}}\\
\shcmd{filter tcp\_sport \param{value}}\\
\shcmd{filter udp\_dport \param{value}}\\
\shcmd{filter udp\_sport \param{value}}

\paramdesc{\param{value}~可以是一个2字节的表示端口的数字,
或者是一个服务名(在/etc/services中定义)。
如果是数字可以是一个以0x开头的16进制数,
也可以是一个以0开头的八进制数,或十进制的数。范围为0到65535。}


使用行模式来显示(这种模式下不会看到包的具体数据)
\begin{lstlisting}
netfmt -N -n -l -1 -f /tmp/nettl_t* | more
\end{lstlisting}

在每行的显示前加上时间戳
\begin{lstlisting}
netfmt -T -n -l -1 -f /tmp/nettl_t* | more
\end{lstlisting}

有关~HP-UX~下抓包的~nettl~和~netfmt~的更多信息参考
\cite{nettl-intro}
\cite{nettl-man}
\cite{netfmt-man}。


\section{nc(netcat)- Linux下调试调试TCP/UDP的工具}

\index{nc}

\subsection{简介}

\shcmd{nc}~是~Linux~提供的一个能用于模拟~TCP/UDP~服务器/客户端的工具。

% TODO

\section{lsof}
\index{lsof}
\subsection{简介}

\shcmd{lsof}~是~Unix~平台下用于检查系统打开文件的工具。
它还能用于检查某个端口是被哪个进程正在使用。
类似的工具有~\shcmd{fuser}\index{fuser},但~\shcmd{lsof}~的功能最强大。

\subsection{用法示例}

\subsubsection{用于检查网络连接信息}

列出所有的网络连接信息:
\begin{lstlisting}
$ sudo lsof -i
\end{lstlisting}

查看使用端口~80~的进程:
\begin{lstlisting}
$ sudo lsof -n -P -i:80
COMMAND   PID     USER   FD   TYPE   DEVICE SIZE/OFF NODE NAME
nginx   24797     root    6u  IPv4 20657646      0t0  TCP *:80 (LISTEN)
nginx   24797     root    7u  IPv6 20657647      0t0  TCP *:80 (LISTEN)
nginx   24798 www-data    6u  IPv4 20657646      0t0  TCP *:80 (LISTEN)
nginx   24798 www-data    7u  IPv6 20657647      0t0  TCP *:80 (LISTEN)
\end{lstlisting}

其中参数~\shcmd{-n}~表示不进行\emph{机器名}的~DNS~解析,
参数~\shcmd{-P}~表示不进行\emph{端口}名字解析。

\subsubsection{检查被删除了之后仍然被应用程序使用的文件}

\begin{lstlisting}
$ sudo lsof | grep deleted
systemd-l   387                             root  txt       REG      179,2     202300      72085 /usr/lib/systemd/systemd-logind (deleted)
systemd   13694                               pi  txt       REG      179,2    1239132      72068 /usr/lib/systemd/systemd (deleted)
(sd-pam)  13695                               pi  txt       REG      179,2    1239132      72068 /usr/lib/systemd/systemd (deleted)
php-cgi   25632                         www-data    3u      REG       0,40          0     524256 /tmp/.ZendSem.7JSJbe (deleted)
php-cgi   25633                         www-data    3u      REG       0,40          0     524256 /tmp/.ZendSem.7JSJbe (deleted)
php-cgi   25634                         www-data    3u      REG       0,40          0     524256 /tm
\end{lstlisting}
