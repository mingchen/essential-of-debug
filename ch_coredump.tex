%# -*- coding: utf-8 -*-

\chapter{Core dump的调试}
\label{ch:coredump}

\section{产生coredump的原因}

core dump\index{coredump}分为主动产生和被动产生的,
比如主动调用\code{abort()}命令就会主动产生coredump。

另一种产生~core dump~的原因是程序收到了一下信号\index{信号}会产生core dump。
会产生~core dump~的信号\footnote{关于信号的更多信息参见man 7 signal。}见表~\ref{sec:tab_core_signals}~所示。

\begin{table}[!hbp]
\caption{产生core dump的信号列表} \label{sec:tab_core_signals}
\begin{tabular}{l|c|l}
\hline
信号名 & 信号值 & 信号描述 \\
\hline
SIGQUIT &    3	&	Quit from keyboard \\
SIGILL  &    4	&	Illegal Instruction \\
SIGABRT &    6	&	Abort signal from abort(3) \\
SIGFPE  &    8	&	Floating point exception \\
SIGSEGV &    11	&	Invalid memory reference \\
SIGBUS  &    10,7,10 &	Bus error (bad memory access) \\
SIGSYS  &    12,-,12 &	Bad argument to routine (SVID) \\
SIGTRAP &    5	&	Trace/breakpoint trap \\
SIGXCPU &    24,24,30 & CPU time limit exceeded (4.2 BSD) \\
SIGXFSZ &    25,25,31 & File size limit exceeded (4.2 BSD) \\
SIGIOT  &    6	&	IOT trap. A synonym for SIGABRT \\
\hline
\end{tabular}
\end{table}

某些信号的\emph{信号值}这一列对应有三个值,因为信号是平台相关的,
在不同的平台上有不同的值。第一个是alpha和sparc平台定义的该信号的值,
第二个是i386,ppc和sh平台定义的该信号的值,
第三个值是mips平台定义该信号的值。
如果是-表示该平台没有实现该信号。

\section{Linux下core dump的配置}
\subsection{控制是否产生core dump}
在Linux下是否产生从core dump文件以及对产生的core dump的大小限制是可配置的,
在Bash/sh/ksh下可以使用ulimit\footnote{csh的命令是limit。}\index{ulimit}命令来查看该设置:\\
\begin{lstlisting}
$ ulimit -c
1000000
\end{lstlisting}
如果输出是0表示不产生core dump文件,
这个时候为了获得core dump就需要修改这个参数,
把它改成一个正数或者unlimited表示对core dump的文件大小没有限制。

\begin{lstlisting}
$ ulimit -c unlimited
$ ulimit -c
unlimited
\end{lstlisting}

\subsection{定制core dump的文件名}

有的Linux默认的core dump的文件名就是core,有的是core.<pid>,
一般core dump的产生于程序的当前运行目录。

对于~Linux 2.5~以上的版本,我们可以通过配置Linux的内核参数
\emph{kernel.core\_pattern}\index{kern.core\_pattern}
来指定core dump的产生位置及文件名,这个文件名是一个模板,
在文件名中可以使用~\%~开头的修饰符来,
在实际产生core dump文件的时候会根据这个修饰符来做相应的替换。
可用的修饰符见表~\ref{sec:tab_core_pattern}~所示。

\begin{table}
\begin{tabular}{r|l}
\hline
通配符 & 含义   \\
\hline
\%\%  &   A single \% character    \\
\%p  &   PID of dumped process   \\
\%u  &   real UID of dumped process  \\
\%g  &   real GID of dumped process  \\
\%s  &   number of signal causing dump   \\
\%t  &   time of dump (secs since 0:00h, 1 Jan 1970) \\
\%h  &   hostname (same as the 'nodename' returned by uname(2))  \\
\%e  &   executable filename \\
\hline
\end{tabular}
\caption{kern.core\_pattern可用的修饰符}\label{sec:tab_core_pattern}
\end{table}

比如如果我们指定core dump文件名的名字模板为
\begin{lstlisting}
/var/crash/%e-%p-%s.core
\end{lstlisting}

如果我们有一个名为test\_fmr,
进程号为24214的的进程因为SIGSEGV信号而异常退出了,
它会在~/var/crash~下生成一个如下一个core dump文件:
\begin{lstlisting}
/var/crash/test_fmr-24214-11.core
\end{lstlisting}

\emph{kern.core\_pattern}~也可以通过proc文件系统来进行配置,
它在/proc文件系统中的位置为:
\begin{lstlisting}
/proc/sys/kernel/core_pattern
\end{lstlisting}

如果是~Linux 2.4~的版本,由于没有内核参数\emph{kernel.core\_pattern},
它对core dump的文件名的控制功能就没有自如了,还好它也有一个内核参数
\emph{kernel.core\_uses\_pid}
\footnote{这个参数在新版本的内核是被遗弃的,不再建议使用。}
\index{kern.core\_uses\_pid},
可以用来控制core生成时是否包含进程的pid的。
它的值可以是0或其它值,
如果是0则产生的core dump的文件简单的命名为core,
否则的话产生的core dump文件格式为core.PID。
\emph{kernel.core\_uses\_pid}在/proc文件系统中的位置为:
\begin{lstlisting}
/proc/sys/kernel/core_uses_pid
\end{lstlisting}

可以通过sysctl、/etc/sysctl.conf或/proc文件系统来修改这两个参数,
只有修改~/etc/sysctl.conf~才会永久生效。

示例:
\begin{lstlisting}
# sysctl -w kernel.core_pattern=/var/crash/%e-%p-%s.core
\end{lstlisting}

修改~/etc/sysctl.conf~后需要执行如下命令来从新读入~/etc/sysctl.conf:
\begin{lstlisting}
# sysctl -p
\end{lstlisting}

我们可以借助valgrind(参见\ref{valgrind},第\pageref{valgrind}页)、
yamd(参见\ref{yamd},第\pageref{yamd}页)等工具来帮助我们发现程序的内存错误。

\subsection{捕获会导致core dump的信号}

下面的代码能捕获并处理SIGSEGV信号:

\lstinputlisting[language={C++},numbers=left]{code/segv_handler.cpp}
\begin{figure}
\caption{捕获会导致core dump的信号的代码}
\end{figure}

链接的时候需要指定~\code{-rdynamic -ldl}~选项,这样才能获得栈上的信息。

\subsection{定位出错的源代码行}
调用栈会给出每一帧的地址,我们需要知道这个地址来获得源代码的对应的
行号来定位出现问题的代码。可以使用~GDB~的~\code{list}~命令来定位,如下所示:
\begin{lstlisting}
$ gdb foo
(gdb) l *0x804904a
0x804904a is in Foo::run() (test_signal.cpp:181).
\end{lstlisting}

另外一种方式就是通过addr2line\index{addr2line}来获得:
%\begin{lstlisting}[language={sh}]
\begin{lstlisting}
$ addr2line -e foo 0x804904a
/home/cm/sandbox/test_signal.cpp:181
\end{lstlisting}

\subsection{C++名字重整}
\index{c++filt}

可以通过这篇文章获取更多的在程序在收到SIGSEGV信号时获取调用栈的信息:\\
Obtaining a stack trace in C upon SIGSEGV\cite{obtain-sigsegv-stack}

\subsection{C++类的内存布局}

针对~GCC~编译后的C++程序,如果一个类存在虚函数,
那么它的内存布局的第一个成员时虚表指针,然后才是其它的成员函数。

参考下面的代码演示了类~\code{Bar}~的内存布局。

\begin{lstlisting}
#include <stdlib.h>

class Foo
{
    public:
        Foo()
        {
            x = 0x12345678;
            y = 0x98765432;
            p = (void*)0x3c3c3c3c;
        }

        virtual void run() { }

        virtual ~Foo() { }

    private:
        int x;
        int y;
        void* p;
};

class Bar : public Foo
{
    public:
        Bar() : b(911) {}

        virtual void run() { }

    private:
        int b;
};

int main()
{
    Bar* f = new Bar;
    abort();
}
\end{lstlisting}

通过~GDB~我们可以清楚的看到~\code{Bar}~的内存布局,
它的第一个成员变量即为虚表指针。

\begin{lstlisting}
$ gdb ./a.out
Program received signal SIGABRT, Aborted.
0x0012d422 in __kernel_vsyscall ()
(gdb) f 3
#3  0x0804860f in main () at memlayout.cpp:37
37          abort();
(gdb) p f
$1 = (Bar *) 0x804b008
(gdb) p *f
$2 = (Bar) {
  <Foo> = {
    _vptr.Foo = 0x8048808,
    x = 305419896,
    y = -1737075662,
    p = 0x3c3c3c3c
  },
  members of Bar:
  b = 911
}
(gdb) x/8x f
0x804b008:      0x08048808      0x12345678      0x98765432      0x3c3c3c3c
0x804b018:      0x0000038f      0x00020fe9      0x00000000      0x00000000
\end{lstlisting}

\section{使用GDB分析core dump文件}

\begin{table}
\begin{tabularx}{400pt}{l|X}
\hline
GDB命令 & 说明 \\
\hline
bt                  &   查看core的调用栈    \\
bt full		        &   查看得完整调用栈信息,包含每一帧的参数值及局部变量的值等 \\
thread apply all bt &   查看所有线程的调用栈 \\
frame $n$           &   切换到调用栈的第~$n$~帧 \\
info threads	    &   查看当前的函数的线程的信息(包括有多少个线程及每个线程的栈顶) \\
info frame 	        &   查看当前帧的信息 \\
info local          &   查看当前函数局部变量的值 \\
info reg		    &   查看寄存器的值 \\
\hline
\end{tabularx}
\caption{GDB调试core dump的常用命令}
\end{table}


\section{调试其它机器上产生的core dump文件}

core dump文件是程序在异常退出时内存的映像。

为了减少core dump文件的大小,core dump文件只保存了可写的节的内存映像,
而没有包含程序的可执行代码(不可变的)。

调试其它机器上产生的core dump文件不仅需要其它导致core dump的可执行程序,
而且还需要它依赖的动态库,否则的话使用bt命令可能看到不是正确的结果。

\subsection{示例}

使用~GDB~的~\code{info shared}\index{info shared}
来获取可执行程序依赖的动态库列表。
然后把这些动态库打包。

在产生core dump的机器上执行把这些动态库打包:

\begin{lstlisting}
$ gdb ust ust-3124.core
(gdb) info shared
From       To         Syms Read Shared Object Library
0x4001e44c 0x40026e80 Yes       /rlib/libpthread.so.0
0x4004d9a0 0x4006adac Yes       /usr/lib/libstdc++-libc6.2-2.so.3
0x4007faa0 0x400966b4 Yes       /rlib/libm.so.6
0x400ba590 0x40197bb4 Yes       /rlib/libc.so.6
0x40001fe0 0x40013808 Yes       /lib/ld-linux.so.2
0x401fa454 0x40200310 Yes       /lib/libnss_files.so.2
0x40203010 0x40204c50 Yes       /lib/libnss_dns.so.2
0x40208c48 0x40212480 Yes       /lib/libresolv.so.2
(gdb) quit
$ cd /
$ tar czf ust_solibs_orig.tar.gz        \
    /rlib/libpthread.so.0               \
    /usr/lib/libstdc++-libc6.2-2.so.3   \
    /rlib/libm.so.6             \
    /rlib/libc.so.6             \
    /lib/ld-linux.so.2          \
    /lib/libnss_files.so.2      \
    /lib/libnss_dns.so.2        \
    /lib/libresolv.so.2
\end{lstlisting}

然后把产生的core dump的可执行文件、
core dump文件以及依赖的动态库拷贝到目标机器上。
使用gdb的~\code{set solib-absolute-prefix}\index{solibs-absolute-prefix}
来设置可执行程序的加载的动态库的绝对路径。


\begin{lstlisting}
$ mkdir /home/cm/ust_solibs_orig
$ cd /home/cm/ust_solibs_orig
$ tar zxf ust_solibs_orig.tar.gz
$ gdb ust
(gdb) set solib-absolute-prefix /home/cm/ust_solibs_orig
(gdb) core ust-3124.core
(gdb) bt
(gdb) bt full
\end{lstlisting}

\section{无法获得调用栈的core dump文件分析}

对于某些情况下的core dump文件,使用\code{bt}命令无法获得完整的调用栈。
这个时候需要借助于反汇编的帮助。使用 \\
\begin{lstlisting}
x/i \$eip \index{eip}
\end{lstlisting}
来查看当前的正在执行的指令,一般来说这个指令会有非法的地址访问。
示例:

使用info来查看当前的寄存器信息:\\
\begin{lstlisting}
info reg
\end{lstlisting}

结合当前的寄存器信息和当前的指令可以看出导致core的原因。

\begin{lstlisting}
Program terminated with signal 11, Segmentation fault.
Cannot access memory at address 0xb7f07000
#0  0x0826db92 in UcpfIdbAttachment::IsOmaNniAttachment (this=Cannot access memory at address 0xb6b04150) at idb/UcpfIdbAttachment.h:605
in idb/UcpfIdbAttachment.h
(gdb) bt
#0  0x0826db92 in UcpfIdbAttachment::IsOmaNniAttachment (this=Cannot access memory at address 0xb6b04150)
    at idb/UcpfIdbAttachment.h:605
Cannot access memory at address 0xb6b04148
(gdb) x/i $eip
0x826db92 <_ZNK17UcpfIdbAttachment18IsOmaNniAttachmentEv+6>:
    mov    0x10b(%eax),%al
(gdb) info reg
eax            0x0          0
ecx            0x89         137
edx            0xb250d218   -1303326184
ebx            0x1caff4     1880052
esp            0xb6b04148   0xb6b04148
ebp            0xb6b04148   0xb6b04148
esi            0x0          0
edi            0xb6b04400   -1229962240
eip            0x826db92    0x826db92
eflags         0x10292      66194
cs             0x73         115
ss             0x7b         123
ds             0xc02e007b   -1070727045
es             0x7b         123
fs             0x0          0
gs             0x33         51
\end{lstlisting}

\begin{lstlisting}
$ objdump -S foo > foo.asm
\end{lstlisting}

\begin{lstlisting}
00000000 <_ZNK17UcpfIdbAttachment18IsOmaNniAttachmentEv>:
   0:   55                      push   %ebp
   1:   89 e5                   mov    %esp,%ebp
   3:   8b 45 08                mov    0x8(%ebp),%eax
   6:   8a 80 0b 01 00 00       mov    0x10b(%eax),%al
\end{lstlisting}


\section{总结}

对于如何定制Linux下产生的core dump的文件名的更多信息请参考~Linux~的~\code{core}的手册页。
