\chapter{内存问题}
\section{简介}
C/C++语言因为没有垃圾回收机制,需要应用程序自身对内存进行管理,
这就可能导致内存管理的混乱。下面列举一些常见的内存相关的问题:

\begin{itemize}
\item 访问已经被释放的内存空间;
\item 释放已经被释放的内存;
\item 重叠的内存操纵,比如memcpy时源地址和目的地址有重叠的情况
\end{itemize}

\section{缓冲区溢出}

\emph{缓冲区溢出}\index{缓冲区溢出}(Buffer overflow)~现在被越来越多的人知道了,
大家都知道它给操作系统及网络安全带来了严重的安全隐患,
在CERT(http://www.cert.org)中报告的问题中\emph{缓冲区溢出}一直是占有最大的比例。

作为一个程序员,如何编写安全的代码,避免缓冲区溢出,
提高程序的健壮性是本节的目的。

\subsection{缓冲区溢出的原理}
\emph{缓冲区溢出}(Buffer overflow)是由于C语言缺乏类型安全引起的,

\subsection{缓冲区溢出的预防}

对C语言来说,在对字符串进行操作时,遵循如下规则:

\begin{itemize}
\item 使用strncpy代替strcpy
\item 使用strncat代替strcat
\item 避免使用puts,scanf
\end{itemize}

\subsubsection{编写安全的代码}

\subsubsection{使用工具预防缓冲区溢出}

TODO: https://www.cert.org/secure-coding/


\section{匿名对象的生命周期}

如下一段代码有一个内存相关的bug:

\begin{lstlisting}[language={C++},numbers=left]
std::ostringstream oss;
...
...
printf("oss: %s\n", s.str().c_str());
\end{lstlisting}

\emph{c\_str()} 产生了一个临时临时变量,在 \emph{printf()} 要使用的时候可能已经从栈上消失了,
程序运行的结果就是可能打印出来的是乱码,因为 \emph{c\_str()} 返回的内存地址已经不是有效的内存地址了。

Book: Thinking in C++


\section{调试内存问题的工具}

TODO
